Well, I've been thinking about a lot of little things lately. Feminism, and the relationships, similarities, and differences of and between men and women.  Also been thinking a lot about money and what it is and what, if anything, it means. And, I have been thinking about how people think, and the how different the thinking processes of living things is from the (non-) thinking processes of computers, and what artificial intelligence might mean, given this.  

It seems to me that humans think about the world, the universe, by comparing it to things they already know.  "What is it like?" is the question. "How is this thing like the things we already know and understand?" Computers "think" by asking: "How are these things different?" "What are the differences between the things?"

From those differences, computers calculate, using any of an infinite variety of algorithms, what the universe is. But, to my mind, computers are really calculating how the universe is different from itself. While humans are figuring out how the universe is like itself.

Which way is better?

Being a human, I tend to come down on the side the way humans do things. I tend to believe that life is best thought of as an indivisible thing. It has parts, which can be described as parts, or as a sort of hologrammatic minimalization of what life, as a whole, is.  Like some fractals (there is that idea again: what is life like?), life is not the sum of its parts. Each of its parts is what it is.
